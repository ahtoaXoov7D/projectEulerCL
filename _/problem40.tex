\documentclass{article}
\begin{document}

\newcount \accum
\newcount \value \newcount \digital \newcount \temp
\newcount \datan \newcount \datai \newcount \datas
\newcount \topof \newcount \topofx \newcount \looper \newcount \looperx

\newcommand{\hgrab}{
  % Pass in argument through \value and number of times to \digital
  % Returns in \value
  \let\gnext=\hgrab
  \ifnum \digital=0
    \let\gnext=\relax
  \else
    \divide \value by 10
    \advance \digital by -1
  \fi
  \gnext
}

\newcommand{\grab}[2]{
  % Returns \value
  \value=#1
  \digital=#2
  \hgrab
}

\newcommand{\hup}{
  % Pass in argument through \value and number of times to \digital
  % Returns in \value
  \let\pnext=\hup
  \ifnum \digital=0
    \let\pnext=\relax
  \else
    \multiply \value by 10
    \advance \digital by -1
  \fi
  \pnext
}

\newcommand{\gohup}[2]{
  % Returns \value
  \value=#1
  \digital=#2
  \hup
}

\newcommand{\helper}{
  % Call as \helper with \datan, \datai, \datas, \topof, \topofx, \looper, \looperx
  % Returns result in \value
  % Where n is the current number, i is the number of remaining indices, s is the buffered string
  % \looperx is the number of digits remaining, \looper is the number of digits starting
  % \topofx is the number of times this digit count remains, \topof is the number starting for \topofx
  \let\hnext=\helper
  \ifnum \looperx=0
    \datas=\datan
    \advance \topofx by -1
    \ifnum \topofx=0
      \multiply \topof by 10
      \topofx=\topof
      \advance \looper by 1
    \fi
    \looperx=\looper
    \advance \datan by 1
  \else
    \advance \looperx by -1
    \grab{\datas}{\looperx}
    \ifnum \datai=0
      \let\hnext=\relax
    \else
      \temp=\value
      \gohup{1}{\looperx}
      \multiply \value by \temp
      \advance \datai by -1
      \advance \datas by -\value
    \fi
  \fi
  \hnext
}

\newcommand{\digitof}[1]{
  \datan=2
  \datai=#1
  \advance \datai by -1 % Use 1-indexing
  \datas=1
  \looperx=1
  \looper=1
  \topof=9
  \topofx=9
  \helper
}

\accum=1
\digitof{1}
\multiply \accum by \value
\digitof{10}
\multiply \accum by \value
\digitof{100}
\multiply \accum by \value
\digitof{1000}
\multiply \accum by \value
\digitof{10000}
\multiply \accum by \value
\digitof{100000}
\multiply \accum by \value
\digitof{1000000}
\multiply \accum by \value

\the\accum

\end{document}
